
\chapter{Elements of mathematics} \label{continuum}

%%==============================================================================
\section{Mean value theorem}

\index{Mean value theorem!Definition}
If the function $f(x)$ is continuous on the \emph{closed interval} $[a,b]$ and is differentiable on the \emph{open interval} $(a,b)$ there exists a $c \in (a,b)$ such that 
\begin{align}
  \label{mean_value_theorem}
  \totder{}{x} f(x) = \frac{f(b) - f(a)}{b - a}.
\end{align}

%%==============================================================================
\section{Fundamental theorem of algebra}

\index{Fundamental theorem of algebra}
Every polynomial equation of degree one or greater has at least one complex root. 

%%==============================================================================
\section{Fundamental theorems of calculus}

\index{Fundamental theorems of calculus!First theorem}
First, if a given function $f(x)$ is continuous on a \emph{closed interval} $[a,b]$ and $F(x)$ is the indefinite integral of $f(x)$ on $[a,b]$, then
\begin{align}
  \label{first_fundamental_theorem_of_calculus}
  \int_a^b f(x) \d{x} = F(b) - F(a).
\end{align}

\index{Fundamental theorems of calculus!Second theorem}
Second, if $f(x)$ is a continuous function on an \emph{open interval} $(a,b)$ with $c \in (a,b)$, the anti-derivative or integral $F(x)$ is defined as
\begin{align}
  F(x) = \int_a^x f(t) \d{t} \hspace{5mm} \text{and} \hspace{5mm} 
  \totder{F(x)}{x} = f(x)
\end{align}
at each point in $(a,b)$.

%%==============================================================================
\section{Fundamental lemma of calculus of variations}

%%==============================================================================
\section{Divergence theorem} \label{ssn_divergence_theorem}

\index{Divergence theorem!Definition}
As first discovered by \citet{lagrange_1762}, the \emph{Divergence theorem} relates integral of the divergence of a continuously-differentiable vector field $\mathbf{j}$ inside a open, bounded volume $\Omega$ to the integral of the outward flux of the vector field across the surface of the volume $\Gamma$,
\begin{align}
  \label{divergence_theorem}
  \int_{\Gamma} \mathbf{j} \cdot \mathbf{n} \d{\Gamma} = \int_{\Omega} \nabla \cdot \mathbf{j} \d{\Omega}.
\end{align}
The proof of this may be found in any multi-dimensional calculus textbook.

%%==============================================================================
\section{Leibniz's rule} \label{ssn_liebniz_rule}

\index{Leibniz's rule!Definition}
Leibniz formula, referred to as \emph{Leibniz's rule for differentiating an integral with respect to a parameter that appears in the integrand and in the limits of integration}, states that
\begin{align}
  \label{leibniz_rule}
  \totder{}{x} \int_{a(x)}^{b(x)} f(x,y) \d{y} = &+ \int_{a(x)}^{b(x)} f_x(x,y) \d{y} \notag \\
  &+ f(x,b(x)) \totder{b}{x} - f(x,a(x))\totder{a}{x},
\end{align}
where $f(x,y)$ and $f_x(x,y)$ are both continuous over the finite domain $y \in [a(x),b(x)]$.

\vspace{2mm}
\noindent \textbf{Proof:}

Let
$$I(x,a,b) = \totder{}{x} \int_{a(x)}^{b(x)} f(x,y) \d{y}.$$
Using the chain rule,
$$\frac{\mathrm{d}I}{\mathrm{d}x} = \parder{I}{x} \totder{x}{x} + \parder{I}{a} \totder{a}{x} + \parder{I}{b} \totder{b}{x},$$
Due to the fact that the integration is performed over the coordinate $y$, the linear operations of partial $x$-differentiation and integration over $y$ may be safely exchanged,
\begin{align*}
  \parder{I}{x} &= \parder{}{x} \int_{a(x)}^{b(x)} f(x,y) \d{y} = \int_{a(x)}^{b(x)} f_x(x,y) \d{y}.
\end{align*}
Next, using first fundamental theorem of calculus definition (\ref{first_fundamental_theorem_of_calculus}), if $F(x,y)$ is the indefinite integral of $f(x,y)$ with respect to $y$ on $y \in [a,b]$,
\begin{align*}
  \parder{I}{a} &= \parder{}{a} \int_{a(x)}^{b(x)} f(x,y) \d{y} \\
                &= \parder{}{a} \Big[ F(x,b(x)) - F(x,a(x)) \Big] = - f(x,a(x)).
\end{align*}
Likewise,
\begin{align*}
  \parder{I}{b} &= \parder{}{b} \int_{a(x)}^{b(x)} f(x,y) \d{y} \\
                &= \parder{}{b} \Big[ F(x,b(x)) - F(x,a(x)) \Big] = f(x,b(x)).
\end{align*}
Therefore, combining the above relations results in
$$\totder{I}{x} = \int_{a(x)}^{b(x)} f_x(x,y) \d{y} + f(x,b(x))\totder{b}{x} - f(x,a(x))\totder{a}{x}. \qed$$

%%==============================================================================
\section{Reynolds transport theorem} \label{ssn_reynolds_transport_theorem}

Within an arbitrary time-evolving volume $\Omega(t) \in \R^3$ with boundary $\Gamma(t)$ moving with the flow of a fluid, \emph{Leibniz's rule in three dimensions} -- better known in continuum mechanics as \index{Reynolds transport theorem} \emph{Reynolds transport theorem} \citep{reynolds_1903} -- states that for a given quantity $\phi$,
\begin{align}
  \label{reynolds_transport_theorem}
  \totder{}{t} \int_{\Omega(t)} \phi \d{\Omega}(t) = \int_{\Omega(t)} \parder{\phi}{t} \d{\Omega(t)} + \int_{\Gamma(t)} \phi \mathbf{u}_b \cdot \mathbf{n} \d{\Gamma(t)},
\end{align}
where $\mathbf{u}_b$ is the velocity of surface $\Gamma(t)$ due to changes in $\Omega(t)$ and $\mathbf{n}$ is the outward-facing unit-normal vector for $\Gamma(t)$.
Note that if the volume remains constant, \ie, $\Omega \neq \Omega(t), \Gamma \neq \Gamma(t)$, $\mathbf{u}_b = 0$, and this relation reduces to
\begin{align}
  \label{constant_volume_reynolds}
  \totder{}{t} \int_{\Omega} \phi \d{\Omega} = \int_{\Omega} \parder{\phi}{t} \d{\Omega}.
\end{align}

%%==============================================================================
\section{Continuity equations} \label{ssn_continuity_equations}

Within a arbitrary fixed volume $\Omega$ with boundary $\Gamma$, 
\begin{align*}
  \begin{matrix}
    \text{the total} \\
    \text{rate of change} \\
    \text{of quantity $\phi$} \\
    \text{in $\Omega$}
  \end{matrix} \hspace{2.5mm} = \hspace{2.5mm} 
  \begin{matrix}
    \text{the inward flux} \\
    \text{of $\phi$ across the} \\
    \text{boundary $\Gamma$} \\
  \end{matrix} \hspace{2.5mm} + \hspace{2.5mm} 
  \begin{matrix}
    \text{the generation} \\
    \text{of quantity $\phi$} \\
    \text{within $\Omega$}
  \end{matrix}, 
\end{align*}
or mathematically, the integral form of the \emph{continuity equation}
\begin{align}
  \label{integral_continuity_equation}
  \totder{}{t} \int_{\Omega} \phi \d{\Omega} \hspace{2.5mm} = \hspace{2.5mm} - \int_{\Gamma} \mathbf{j} \cdot \mathbf{n} \d{\Gamma} \hspace{2.5mm} + \hspace{2.5mm} \int_{\Omega} f \d{\Omega},
\end{align}
with source term $f$, material velocity $\mathbf{u}$, and outward-pointing unit-normal-vector $\mathbf{n}$.  Applying Reynolds tranport theorem corollary (\ref{constant_volume_reynolds}) to the total time derivative and divergence theorem (\ref{divergence_theorem}) \index{Divergence theorem!Regarding continuity equations} to the surface integral,
\begin{align*}
  \int_{\Omega} \parder{\phi}{t} \d{\Omega} + \int_{\Omega} \nabla \cdot \mathbf{j} \d{\Omega} &= \int_{\Omega} f \d{\Omega}.
\end{align*}
Therefore, because the integral was taken arbitrarily, the \emph{differential form} of the continuity equation is
\begin{align}
  \label{differential_continuity_equation}
  \parder{\phi}{t} + \nabla \cdot \mathbf{j} = f.
\end{align}

In the case that the flux is composed of advective $\phi \mathbf{u}$ and non-advective $\bm{\psi}$ terms, \ie $\mathbf{j} = \phi \mathbf{u} + \bm{\psi}$, 
\begin{align}
  \label{conservative_continuity_equation}
  \parder{\phi}{t} + \nabla \cdot \left( \phi \mathbf{u} \right) + \nabla \cdot \bm{\psi} = f,
\end{align}
Next, the advection flux divergence term is expanded so that
\begin{align*}
  \parder{\phi}{t} + \mathbf{u} \cdot \nabla \phi + \phi \nabla \cdot \mathbf{u} + \nabla \cdot \bm{\psi} = f,
\end{align*}
which after using the material derivative becomes
\begin{align}
  \label{nonconservative_continuity_equation}
  \totder{\phi}{t} + \phi \nabla \cdot \mathbf{u} + \nabla \cdot \bm{\psi} = f.
\end{align}
In context of continuum mechanics, Equations (\ref{conservative_continuity_equation}) and (\ref{nonconservative_continuity_equation}) are referred to as the \emph{conservative} and \emph{non-conservative} forms of the continuity equation, respectively \citep{anderson_1995}.

%%==============================================================================
\section{Conservation equations} \label{ssn_conservation_equations}

The conservation of mass equation i

%%==============================================================================
\section{Finite-volume method} \label{ssn_finite_volume_method}

\index{Finite-volume method}
Let $V$ be the volume of the domain $\Omega$ and $\bar{g} = V^{-1} \int_{\Omega} g \d{\Omega}$ be the volume average of a measureable quantity $g(x,t)$.  Integral continuity equation (\ref{integral_continuity_equation}) may be represented exactly as
\begin{align}
  \label{finite_volume_continuity}
  V \totder{\bar{\phi}}{t} + \int_{\Gamma} \mathbf{j} \cdot \mathbf{n} \d{\Gamma} &= V \bar{f}.
\end{align}
This is the fundamental equation from which the class of numerical approximation methods termed \emph{finite-volume methods} emerged.  These methods produce a local approximation of the average of a quantity over finite cells, and require a specification of the cell flux $\mathbf{j}$.  Note that similar to the derivation of continuity equation (\ref{differential_continuity_equation}), the surface integral may be converted to a volume integral using divergence theorem (\ref{divergence_theorem}),
\begin{align}
  \label{finite_volume_continuity}
  V \totder{\bar{\phi}}{t} + \int_{\Omega} \nabla \cdot \mathbf{j} \d{\Omega} &= V \bar{f}.
\end{align}
